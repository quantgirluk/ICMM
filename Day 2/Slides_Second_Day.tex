\documentclass[8pt, hyperref={pdfpagelabels=false}]{beamer}
\usetheme{Boadilla} %{Pittsburgh}
% For more themes: https://deic.uab.cat/~iblanes/beamer_gallery/individual/Boadilla-default-default.html


% Packages
\usepackage{lmodern}
\usepackage{tikz}
\usepackage{nicefrac}
\usepackage{upgreek}
\usepackage{amsmath}
\usepackage{relsize}
\usepackage[style=british]{csquotes}
\usepackage[spanish, english]{babel}
\usepackage[utf8]{inputenc}
\usepackage{smartdiagram}
\usepackage{graphicx, epsfig}
\usepackage[T1]{fontenc}
\usepackage{graphicx}
\usepackage{amssymb,amsmath}  % use mathematical symbols
\usepackage{metalogo}
\usepackage[export]{adjustbox}
\usepackage{tabularx}
\usepackage{algorithm,algorithmic}
\usepackage{xcolor}
%\usepackage{emoji}
%\beamerdefaultoverlayspecification{<+->}




\definecolor{links}{HTML}{009B55}
\hypersetup{colorlinks,linkcolor=,urlcolor=links}


\usefonttheme{serif}   % european characters
\usepackage{palatino}         % use palatino as the default font
\usepackage{fontawesome}
\setbeamertemplate{footline}[frame number]
%\definecolor{mycolor}{RGB}{35, 166, 164}

\definecolor{mycolor}{rgb}{0.2, 0.20, 0.80}
\usecolortheme[named=mycolor]{structure}

%Beamer settings
\setcounter{secnumdepth}{0} % sections are level 1
\setbeamercolor{alerted text}{fg=magenta}
%\setbeamerfont{alerted text}{series=\bfseries}
\setbeamertemplate{frametitle}[default][left]
\setbeamertemplate{enumerate item}[square]
\setbeamertemplate{itemize item}[circle]
\setbeamertemplate{sections/subsections in toc}[sections numbered]


%\AtBeginSection[]{
	%	\begin{frame}
		%		\vfill
		%		\centering
		%		\begin{beamercolorbox}[sep=8pt,center,shadow=true,rounded=true]{title}
			%			\usebeamerfont{title}\insertsectionhead\par%
			%		\end{beamercolorbox}
		%		\vfill
		%	\end{frame}
	%}

%\setbeamertemplate{subsection page}
%{
	%	\begingroup
	%	\begin{beamercolorbox}[sep=12pt,center]{section title}
		%		\usebeamerfont{section title}\insertsection\par
		%	\end{beamercolorbox}
	%	\vspace*{10pt}
	%	\begin{beamercolorbox}[sep=8pt,center]{subsection title}
		%		\usebeamerfont{subsection title}\insertsubsection\par
		%	\end{beamercolorbox}
	%	\endgroup
	%}

%\makeatletter
\setbeamertemplate{footline}{
	\leavevmode%
	\hbox{%
		\begin{beamercolorbox}[wd=.333333\paperwidth,ht=2.25ex,dp=1ex,center]{author in head/foot}%
			\usebeamerfont{author in head/foot}\insertshortauthor
		\end{beamercolorbox}%
		\begin{beamercolorbox}[wd=.333333\paperwidth,ht=2.25ex,dp=1ex,center]{title in head/foot}%
			\usebeamerfont{title in head/foot}\insertshorttitle
		\end{beamercolorbox}%
		\begin{beamercolorbox}[wd=.333333\paperwidth,ht=2.25ex,dp=1ex,right]{date in head/foot}%
			\usebeamerfont{date in head/foot}\insertshortdate{}\hspace*{2em}
			\insertframenumber{} / \inserttotalframenumber\hspace*{2ex}
	\end{beamercolorbox}}%
	\vskip0pt%
}
\addtobeamertemplate{footnote}{}{\vspace{2ex}}

%\addtobeamertemplate{background canvas}{\transfade[duration=1]}{}
%My newcommands
\newcommand{\MSE}{\mathrm{MSE}}
\newcommand{\Bias}{\mathrm{Bias}}
\newcommand{\Tra}{\mathrm{Tr}}
\newcommand{\PRIAL}{\mathrm{PRIAL}}
\newcommand{\diag}{\mathrm{diag}}
\newcommand{\ud}{\mathrm{d}}
\newcommand{\Var}{\mathrm{Var}}
\newcommand{\cov}{\mathrm{cov}}
\newcommand{\defby}{\triangleq}
\newcommand{\Yr}{\,\mathrm{Y}}
\newcommand{\E}{\mathbb{E}}


%% Title, author, etc.
\title[UTM]{A Short Introduction to Monte Carlo Methods \\ in  Financial Mathematics}
\subtitle{Workshop}
\author[DSR]{Dialid Santiago\\
	VP Quantitative Strategists at Bank of America\footnote{This talk represents the views of the author alone, and not the views of BofA Securities, Inc., Citigroup,  or any of her previous employers.}\\
	\faTwitter \,\, @Quant\_Girl \\
	\href{https://quantgirl.blog}{\color{purple}quantgirl.blog}\\
	\href{https://github.com/quantgirluk}{https://github.com/quantgirluk}
}
\institute[]{}
\date{October, 2022}
\addtobeamertemplate{title page}{}{\begin{center}
		5th International Conference on Mathematical Modelling\\
		Universidad Tecnológica de la Mixteca 
\end{center}}



% Slides
\begin{document}
	
	
	\frame{\titlepage}
	\frame{\frametitle{Contents}
		\tableofcontents[ ] %subsectionstyle=hide]
	}
	
	
	
	\section{Welcome}
	\frame{
		\frametitle{Objectives}
		
		
		For the short course:
		
		\begin{itemize}
			\item Get you interested on financial mathematics and maybe on pursuing a career as a Quant after graduating or after postgraduate education
			\item Show you the type of problems we encounter so you can get a flavour of the job that quants do everyday
			\item Show you what kind of mathematical tools are required
		\end{itemize}
		
		
		For today: 
		
		\begin{itemize}
			\item Review the Black-Scholes Model for option pricing
			\item Review the definition of Monte Carlo 
			\item See how these concepts are translated into code in Python
			\item Calculate the price of an Asian option using Monte Carlo
		\end{itemize}
	}


\section{Black and Scholes Model}

\frame{
	\frametitle{Options}
	\begin{block}{European Options}
		
		
		
		\begin{itemize}
			
			\item A \alert{European Option} is a contract which conveys to its owner the right, but not the obligation, to buy (\alert{Call}) or sell (\alert{Put}) an \alert{underlying asset or instrument}  at a specified \alert{strike} price \alert{$K$} on a  \alert{expiry} date \alert{$T$}.
			
			\item The \alert{payoff} is given by
			
			$$Call \ \ Payoff = (S_T - K)^{+}$$
			
			$$Put \ \  Payoff = (K - S_T)^{+},$$
			
			where $K$ is the strike price, and \alert{$S_T$} is the price of the underlying asset at expiry.
			
		\end{itemize}
		
	\end{block}
	
	Note: The key difference between American and European options relates to when the options can be exercised: An American option may be exercised either at epiry or at any time before it.
	
}


\frame{
	\frametitle{Black-Scholes Model}
	\begin{block}{Assumptions }
		
		\begin{enumerate}
			\item The \href{https://en.wikipedia.org/wiki/Risk-free_rate}{risk-free interest rate $r$} is known and constant through time
			\item  The stock price $S$ follows a \alert{geometric Brownian motion} with constant parameters $\mu$ and $\sigma$
			\item  Stock pays \alert{no} \href{https://en.wikipedia.org/wiki/Dividend}{dividends}
			\item The option can only be exercised at expiration i.e. it is of \alert{European} type
			\item There are \alert{no transaction costs}
			\item  \alert{Fractional trading} is possible
		\end{enumerate}		
	\end{block}	
}


\frame{
	\frametitle{Black-Scholes Equation}
	
	The idea is to construct a \href{https://en.wikipedia.org/wiki/Self-financing_portfolio}{self-financing portfolio} with an option and \alert{$\Delta$ units of the underlying stock}, i.e. 
	
	\color{blue}
	$$\Pi_t = V_t + \Delta S_t, \qquad t\geq 0,$$
	
	\color{black}
	which satisfies the following two conditions:
	
	\begin{itemize}
		\item 	The portfolio is riskless
		\item  The portfolio earns the risk free rate
	\end{itemize}
	Under these assumptions, we obtain a second order parabollic partial differential equation (PDE) with boundary condition
	\color{blue}{
	\begin{align*}
		\frac{\partial V}{\partial t}  + r^d S_t\frac{\partial V}{\partial S} + \frac{1}{2} \sigma^2 S^2  \frac{\partial^2 V}{\partial S^2}    - r V_t  &= 0, \qquad \qquad t \in [0,T]\\
		V_T =  (S_T - &  K)^{+} .
	\end{align*}
}
\color{black}
	We will see the full derivation inthe Jupyter Notebook.
}


\frame{
\frametitle{Solution or Black-Scholes Formula}

The solution, which  is known as \alert{Black and Scholes formula}, is given by

\color{blue}
\begin{align*}
	V(S_t, t) = \Phi(d_1) S_t - \Phi(d_2)K e^{-r(T-t)},
\end{align*}
\color{black}
where
\color{blue}
\begin{align*}
	d_1 & = \dfrac{1}{\sigma \sqrt{T-t}} \left[ \ln \left(\frac{S_t}{K} \right) + \left( r + \dfrac{\sigma^2}{2}\right) (T-t)  \right]\\
	d_2 & = d_1 - \sigma\sqrt{T-t},
\end{align*}
\color{black}
and $\Phi$ denotes the cumulative distribution function (cdf) of a standard normal distribution.
}

\frame{
	\begin{center}
		\large  \href{https://github.com/quantgirluk/ICMM/blob/main/Day\%202/Part_2_1.ipynb}{Python Time}
	\end{center}
}


\frame{
	\frametitle{Monte Carlo 101}
	\begin{block}{Monte Carlo Integration}
		Let $X$ be either a discrete random variable taking values in a countable or finite set $\Omega$, with probability mass function (p.m.f.) $f_X$, or a continuous random variable taking values in $\Omega = \mathbb{R}^d$, with probability density function (p.d.f) $f_X$. Consider
		
		\begin{equation}
			\theta = \mathbb{E}[\phi (X)] = \begin{cases}
				\sum_{x\in\Omega} \phi(x) f_X(x) & \mbox{if X is discrete} \\
				\int_{\mathbb{R}^d} \phi(x) f_x(x)dx & \mbox{if X is continuous},
			\end{cases}
		\end{equation}
		where $\phi:\Omega\rightarrow\mathbb{R}$. Let $X_1, \cdots, X_n$ be i.i.d. random variables with p.m.f. (or p.d.f. in the continuous case) $f_X$. Then
		\begin{equation}
			\hat{\theta} _n= \dfrac{1}{n}\sum_{i=1}^{n} \phi(X_i),
		\end{equation}
		is called the \alert{Monte Carlo estimator} of the expectation $\theta$.
	\end{block}
	
}


\frame{\frametitle{Monte Carlo 101}
	Monte Carlo methods can be thought of as a stochastic way to approximate integrals.
	
	\begin{algorithm}[H]
		\begin{algorithmic}[1]
			\STATE Simulate independent $X_1, \cdots, X_n$  from a random variable with distribution $f_X$
			\STATE Return $	\hat{\theta} _n= \frac{1}{n}\sum_{i=1}^{n} \phi(X_i)$.
		\end{algorithmic}
		\caption{Monte Carlo}
		\label{alg:seq}
	\end{algorithm}
}


\frame{
	\frametitle{Monte Carlo 101}
	\begin{block}{Monte Carlo Estimate Properties}
		If $\theta = \mathbb{E}[\phi (X)]$ exists, then the Monte Carlo estimator $\hat{\theta}_n$ has the following properties
		\begin{itemize}
			\item \alert{Unbiasedness}
			\begin{equation*}
				\mathbb{E}[\hat{\theta}_n] = \theta ,
			\end{equation*}
			\item \alert{Strong consistency}
			\begin{equation*}
				\hat{\theta}_n \rightarrow  \theta  \mbox{ almost surely as } n\rightarrow \infty.
			\end{equation*}
		\end{itemize}
		
		\textbf{\textit{Proof: }} Note that the linerity of ht expectation implies that
		\begin{equation*}
			\mathbb{E}[\hat{\theta}_n] = \mathbb{E} \left[\dfrac{1}{n}\sum_{i=1}^{n} \phi(X_i)\right] =  \dfrac{1}{n}\sum_{i=1}^{n}\mathbb{E} \left[ \phi(X_i)\right] =\theta.
		\end{equation*}
		Strong consistency is a consequence of the strong law of large numbers applied to $Y_i = \phi(X_i)$.
	\end{block}
	
}


\frame{
	\frametitle{Monte Carlo 101}
	\begin{block}{How Good is the Estimate?}
		\begin{itemize}
			\item Chebyshev's inequality implies
			\begin{equation}
				\mathbb{P}\left(|  \hat{\theta}_n -   \theta |  > c\dfrac{\sigma}{\sqrt{n}}  \right) \leq   \dfrac{n \mathbb{V} [\hat{\theta}_n] }{c^2\sigma^2} =  \dfrac{1}{c^2}
			\end{equation}
			\item The Central Limit Theorem implies that for large $n$
			\begin{equation*}
				\dfrac{\sqrt{n}}{\sigma}( \hat{\theta}_n -   \theta) \sim \mathcal{N}(0,1),
			\end{equation*}
			and then
			\begin{equation*}
				\mathbb{P}\left(|  \hat{\theta}_n -   \theta |  > c\dfrac{\sigma}{\sqrt{n}}  \right) \approx 2 (1-\Phi(c)).
			\end{equation*}
			Hence, by choosing $c_{\alpha}$  such that  $2 (1-\Phi(c)) = \alpha$, we obtain an approximate  $(1-\alpha) 100\%$ confidence interval for $\theta$ as follows
			\begin{equation*}
				\left( \hat{\theta}_n  \pm  c_{\alpha} \dfrac{\sigma}{\sqrt{n}} \right) \approx \left( \hat{\theta}_n  \pm  c_{\alpha} \dfrac{S_{\phi(X)}}{\sqrt{n}} \right)
			\end{equation*}
		\end{itemize}
	\end{block}
}


\frame{
	\begin{center}
		\large  \href{https://github.com/quantgirluk/ICMM/blob/main/Day\%202/Part_2_2.ipynb}{Python time}
	\end{center}
}

\section{Final Comments}

\frame{
\frametitle{Monte Carlo in Action }
	Why do we need Monte Carlo if we have a closed formula in the B-S model?
\begin{itemize}
	\item Multi-asset instruments
	\item More complex products such as \href{https://wayback.archive-it.org/all/20090320135332/http://people.math.jussieu.fr/~boka/enseignement/isifar/refs/Ref_Asiatiques_Rogers_Shi_95.pdf}{Asian options}
	\item Going beyond the Black-Scholes simple model. See for example the \href{https://en.wikipedia.org/wiki/Short-rate_model}{various models} for interest rates.
\end{itemize}
}

	
\frame{
	\begin{center}
		Many thanks for your attention!
	\end{center}
}


\end{document}
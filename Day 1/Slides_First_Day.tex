\documentclass[8pt, hyperref={pdfpagelabels=false}]{beamer}
\usetheme{Boadilla} %{Pittsburgh}
% For more themes: https://deic.uab.cat/~iblanes/beamer_gallery/individual/Boadilla-default-default.html


% Packages
\usepackage{lmodern}
\usepackage{tikz}
\usepackage{nicefrac}
\usepackage{upgreek}
\usepackage{amsmath}
\usepackage{relsize}
\usepackage[style=british]{csquotes}
\usepackage[spanish, english]{babel}
\usepackage[utf8]{inputenc}
\usepackage{smartdiagram}
\usepackage{graphicx, epsfig}
\usepackage[T1]{fontenc}
\usepackage{graphicx}
\usepackage{amssymb,amsmath}  % use mathematical symbols
\usepackage{metalogo}
\usepackage[export]{adjustbox}
\usepackage{tabularx}
\usepackage{algorithm,algorithmic}
\usepackage{xcolor}
%\usepackage{emoji}
%\beamerdefaultoverlayspecification{<+->}




\definecolor{links}{HTML}{009B55}
\hypersetup{colorlinks,linkcolor=,urlcolor=links}


\usefonttheme{serif}   % european characters
\usepackage{palatino}         % use palatino as the default font
\usepackage{fontawesome}
\setbeamertemplate{footline}[frame number]
%\definecolor{mycolor}{RGB}{35, 166, 164}

\definecolor{mycolor}{rgb}{0.2, 0.20, 0.80}
\usecolortheme[named=mycolor]{structure}

%Beamer settings
\setcounter{secnumdepth}{0} % sections are level 1
\setbeamercolor{alerted text}{fg=magenta}
%\setbeamerfont{alerted text}{series=\bfseries}
\setbeamertemplate{frametitle}[default][left]
\setbeamertemplate{enumerate item}[square]
\setbeamertemplate{itemize item}[circle]
\setbeamertemplate{sections/subsections in toc}[sections numbered]


%\AtBeginSection[]{
%	\begin{frame}
%		\vfill
%		\centering
%		\begin{beamercolorbox}[sep=8pt,center,shadow=true,rounded=true]{title}
%			\usebeamerfont{title}\insertsectionhead\par%
%		\end{beamercolorbox}
%		\vfill
%	\end{frame}
%}

%\setbeamertemplate{subsection page}
%{
%	\begingroup
%	\begin{beamercolorbox}[sep=12pt,center]{section title}
%		\usebeamerfont{section title}\insertsection\par
%	\end{beamercolorbox}
%	\vspace*{10pt}
%	\begin{beamercolorbox}[sep=8pt,center]{subsection title}
%		\usebeamerfont{subsection title}\insertsubsection\par
%	\end{beamercolorbox}
%	\endgroup
%}

%\makeatletter
\setbeamertemplate{footline}{
	\leavevmode%
	\hbox{%
		\begin{beamercolorbox}[wd=.333333\paperwidth,ht=2.25ex,dp=1ex,center]{author in head/foot}%
			\usebeamerfont{author in head/foot}\insertshortauthor
		\end{beamercolorbox}%
		\begin{beamercolorbox}[wd=.333333\paperwidth,ht=2.25ex,dp=1ex,center]{title in head/foot}%
			\usebeamerfont{title in head/foot}\insertshorttitle
		\end{beamercolorbox}%
		\begin{beamercolorbox}[wd=.333333\paperwidth,ht=2.25ex,dp=1ex,right]{date in head/foot}%
			\usebeamerfont{date in head/foot}\insertshortdate{}\hspace*{2em}
			\insertframenumber{} / \inserttotalframenumber\hspace*{2ex}
	\end{beamercolorbox}}%
	\vskip0pt%
}
\addtobeamertemplate{footnote}{}{\vspace{2ex}}

%\addtobeamertemplate{background canvas}{\transfade[duration=1]}{}
%My newcommands
\newcommand{\MSE}{\mathrm{MSE}}
\newcommand{\Bias}{\mathrm{Bias}}
\newcommand{\Tra}{\mathrm{Tr}}
\newcommand{\PRIAL}{\mathrm{PRIAL}}
\newcommand{\diag}{\mathrm{diag}}
\newcommand{\ud}{\mathrm{d}}
\newcommand{\Var}{\mathrm{Var}}
\newcommand{\cov}{\mathrm{cov}}
\newcommand{\defby}{\triangleq}
\newcommand{\Yr}{\,\mathrm{Y}}
\newcommand{\E}{\mathbb{E}}


%% Title, author, etc.
\title[ICMM]{A Short Introduction to Monte Carlo Methods \\ in  Financial Mathematics}
\subtitle{Workshop Part 1}
\author[DSR]{Dialid Santiago\\
	VP Quantitative Strategist at Bank of America\footnote{This talk represents the views of the author alone, and not the views of BofA Securities, Inc., Citigroup,  or any of her previous employers.}\\
	\faTwitter \,\, @Quant\_Girl \\
	\href{https://quantgirl.blog}{\color{purple}quantgirl.blog}\\
	\href{https://github.com/quantgirluk}{https://github.com/quantgirluk/ICMM}
}
\institute[]{}
\date{October, 2022}
\addtobeamertemplate{title page}{}{\begin{center}
		5th International Conference on Mathematical Modelling\\
		Universidad Tecnológica de la Mixteca 
\end{center}}



% Slides
\begin{document}
	
	
\frame{\titlepage}
\frame{\frametitle{Contents}
	\tableofcontents[ ] %subsectionstyle=hide]
}



\section{Welcome}
\frame{
	\frametitle{Objectives}
	
	
For the short course:

\begin{itemize}
	\item Get you interested on financial mathematics and maybe on pursuing a career as a Quant after graduating or after postgraduate education
	\item Show you the type of problems we encounter so you can get a flavour of the job that quants do everyday. In particular, we aim to price an European option under the Black-Scholes model by two methods: obtaining the analytical formula, and using Monte Carlo
	\item Show you what kind of mathematical tools are required and illustrate how these concepts are translated into code in Python
\end{itemize}
	

For today: 
	
\begin{itemize}
	\item Recall  concepts on Probaility and Stochastic Processes Theory
	\begin{itemize}
		\item  Random Variables, Stochastic Processes 
		\item  Brownian Motion
		\item Ito's formula 
		\item Geometric Brownian Motion
	\end{itemize}
  \item See how these concepts are translated into code in Python
  \item Take a first look at financial time series in Python
\end{itemize}
}



\section{Basic Conceps}

\frame{
	\frametitle{Basic Concepts}
	\begin{block}{Random Variables}
	Let $(\Omega, \mathcal{F}, \mathbb{P})$ be a \href{https://en.wikipedia.org/wiki/Probability_space}{probability space} and $(S,\Sigma) $ a \href{https://en.wikipedia.org/wiki/Measurable_space}{measurable space}. Then, an $(S, \Sigma)$ \alert{random variable} is a \href{https://en.wikipedia.org/wiki/Measurable_function}{measurable function}
	$$X: \Omega \rightarrow S,$$
	which means that, for every subset $B\in\Sigma $, its pre-image is $\mathcal{F}-$measurable, i.e.;
	$$X^{-1}(B) \in \mathcal{F},$$
	where 
	$$X^{-1}(B) = \{   \omega : X(\omega )\in B \}.$$ 
	\end{block}

\begin{itemize}
	\item  Tipically, $S = \mathbb{R}^d$ for some $d\geq 1$,  and $\Sigma = \mathcal{B}(\mathbb{R}^d)$ is the  corresponding Borel sigma-algebra. 
	\item Examples
		\begin{itemize}
		\item  Discrete random variables: \alert{Bernoulli, Binomial, Poisson}
		\item Continuous random variables: \alert{Uniform, Gaussian, Log-normal, t-Student} 
	\end{itemize}
\end{itemize}
}

\frame{
\begin{center}
\large	\href{https://github.com/quantgirluk/ICMM/blob/bf94e8aa6770c2909ab715dc4ef4b4d3db1e554c/Day\%201/Part\_1\_1.ipynb}{Python Time}
\end{center}
}



\frame{
	\frametitle{Stochastic Processes}
	\begin{block}{Stochastic Process}
For  given \href{https://en.wikipedia.org/wiki/Probability_space}{probability space}   $(\Omega ,{\mathcal {F}},P)$ and \href{https://en.wikipedia.org/wiki/Measurable_space}{measurable space} $(S, \Sigma)$, a \alert{stochastic process} is a collection of $S$-valued random variables 
$$\{ X_t : t \in I\},$$
where the set $I$ is called index set.
	\end{block}
	
	\begin{itemize}
		\item  Tipically, $S = \mathbb{R}^d$ for some $d\geq 1$ and  $\Sigma = \mathcal{B}(\mathbb{R}^d)$ is the  corresponding Borel sigma-algebra
	    \item The index set can be discrete, e.g.  $I=\mathbb{N}$, or continuous e.g. $I=[0,T]$ for some $T\geq 0$.
	\end{itemize}
}

\section{Stochastic Processes}

\frame{
	\frametitle{Stochastic Processes}
	\begin{block}{Brownian Motion or Wiener process}
	A \alert{standard Brownian motion}, or \href{https://en.wikipedia.org/wiki/Wiener_process}{Wiener process},  is a stochastic process $\{ W_t : t \geq 0 \}$ charaterised by the following four properties:
	\begin{enumerate}
		\item $W_0 = 0$
		\item $W_t$ has independent increments
		\item $W_t-W_s\sim \mathcal{N}(0, t-s)$ for any $0\leq s\leq t$ 
		\item $W_t$ is almost surely continuous
	\end{enumerate}
	\end{block}

Here $\mathcal{N}(\mu, \sigma^2)$ denotes the normal or Gaussian distribution with  given mean $\mu$ and variance $\sigma$.
}


\frame{
	\begin{center}
		\large \href{https://github.com/quantgirluk/ICMM/blob/bf94e8aa6770c2909ab715dc4ef4b4d3db1e554c/Day\%201/Part_1_2.ipynb}{Python Time}
	\end{center}
}


\frame{
	\frametitle{Stochastic Processes}
	\begin{block}{Ito's Lemma}
	Suppose that $X =\{X_t : t\geq 0 \}$ is a stochastic \alert{process} which satisfies the following \alert{stochastic differential equation (SDE)}
	\begin{equation*}
		dX_t = \mu(t, X_t )dt + \sigma(t, X_t)dW_t, \qquad t\geq 0,
	\end{equation*}
i.e. 
	\begin{equation*}
	X_t = X_0 +  \int_0^t\mu(s, X_s )ds + \int_0^t \sigma(s, X_s)dW_s, \qquad t\geq 0,
\end{equation*}
where $W$ denotes a standard Brownian motion. Let \alert{$f(t,X)$ be a twice differentiable function}. Then the process  $Y = \{Y_t = f(t, X_t), t\geq 0\}$ satisfies the following SDE
\begin{eqnarray*}
 df(t, X_t) = \left(  \dfrac{\partial f}{\partial t} +  \mu_t \dfrac{\partial f }{\partial x } + \dfrac{1}{2} \sigma_t^2  \dfrac{\partial^2 f}{\partial x^2} \right) dt + \sigma_t \dfrac{\partial f}{\partial x} dW_t,
\end{eqnarray*}
i.e. 
\begin{eqnarray*}
	f(t, X_t) =  f(0, X_0) +  \int_0^t \left(  \dfrac{\partial f}{\partial s} +  \mu_s \dfrac{\partial f }{\partial x } + \dfrac{1}{2} \sigma_s^2  \dfrac{\partial^2 f}{\partial x^2} \right) ds + \int_0^t \sigma_s \dfrac{\partial f}{\partial x} dW_s, \qquad t\geq 0.
\end{eqnarray*}
	\end{block}
	
}


\frame{
	\frametitle{Stochastic Processes}
	\begin{block}{Geometric Brownian Motion}
A \alert{geometric Brownian motion} is a stochastic process defined by the following SDE
\begin{align}
	dS_t &= \mu S_t dt + \sigma S_t W_t, \quad t >0, 
\end{align}
where $S_0 =s_0>0$, $\mu\in\mathbb{R}$, $\sigma>0$, are constants;  and $W$ is a standard Brownian motion. 
	\end{block}

\textbf{\textit{Solution:}}
Let us set a new process as  $X_t = log(S_t)$. Using Ito's formula, we obtain

$$X_t = X_0 +  \left( \mu-\frac{1}{2}\sigma^2 \right)t  + \sigma W_t,$$

or equivalently 

$$log(S_t) = log(s_0) +  \left( \mu-\frac{1}{2}\sigma^2 \right)t  + \sigma W_t.$$


Note that the last expression implies that $\log(S_t)$ follows a \alert{normal distribution} $\mathcal{N}\left(\log(s_0) +  \left( \mu-\frac{1}{2}\sigma^2 \right)t, \sigma^2 t\right )$. This, in turn  implies that 

$$S_t = s_0 \exp\left\{ \left( \mu-\frac{1}{2}\sigma^2 \right)t  + \sigma W_t   \right\}, \quad \forall t>0,$$

follows a \alert{log-normal distribution}. 
}

\frame{
	\begin{center}
		\large\href{https://github.com/quantgirluk/ICMM/blob/bf94e8aa6770c2909ab715dc4ef4b4d3db1e554c/Day\%201/Part_1_3.ipynb}{Python Time } 
	\end{center}
}


\frame{
		\begin{center}
	To finish this session let's take a look at financial time series in \href{https://github.com/quantgirluk/ICMM/blob/bf94e8aa6770c2909ab715dc4ef4b4d3db1e554c/Day\%201/Part_1_4.ipynb}{Python}
	\end{center}
}

\frame{
	\begin{center}
		Many thanks for your attention\\
		See you tomorrow!
	\end{center}
}


\end{document}
